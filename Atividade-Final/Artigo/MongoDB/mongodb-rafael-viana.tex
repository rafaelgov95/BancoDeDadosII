\documentclass[12pt]{article}
\usepackage{multicol}
\usepackage{sbc-template}
\usepackage{subfigure} 
\usepackage{float} 
\usepackage{graphicx,url}
\usepackage[brazil]{babel}   
%\usepackage[latin1]{inputenc}  
\usepackage[utf8]{inputenc}  
% UTF-8 encoding is recommended by ShareLaTex

     
\sloppy

\title{MongoDB Conceitos Básicos}

\author{Rafael Gonçalves de Oliveira Viana\inst{1} }


\address{Sistemas de Informação -- Universidade Federal do Mato Grosso do Sul
	(UFMS)\\
  	Caixa Postal 79400-000 -- Coxim -- MS -- Brazil
  \email{rafael.viana@aluno.ufms.br}
}

\begin{document} 

\maketitle

\begin{abstract}
	Nowaday, with the spread of the Internet, a major challenge in the area of computing, the large-scale data manipulation "Big Data", with this new trend, traditinal SGBDs can no longer offer performace storage.
	
	Because of this and a host of other issues, a new set of tool platforms
	Big Data, called NoSQL (not just SQL), was developed, this article will focus on presenting one of these MongoDB technologies.
\end{abstract}
     
\begin{resumo} 
  Atualmente com a disseminação da Internet crio-se um grande desafio na área de computação, a manipulação de dados em larga escala "Big Data", com essa nova tendência os SGBDs tradicionais não são mais capazes, de prover armazenamento com desempenho.
  
  Por causa desse problema, e outros demais, um novo conjunto de plataformas de ferramentas voltadas para
  Big Data, denominado NoSQL (Not only SQL), tem sido desenvolvida, este artigo estara focado a apresentação de uma dessas tecnologias o MongoDB.
  
  
\end{resumo}


\section{Introdução}
MongoDB promove diversas soluções inovadoras de armazenamento e processamento de grande volume de dados. Estas soluções foram inicialmente criadas para solucionar problemas gerados por aplicações, com grande volume de dados, tenham uma arquitetura que “escale” com grande facilidade de forma horizontal, além da necessidade de persistência dos dados em aplicações nas nuvens (cloud computing). Para atender essas necessidades, grandes empresas como o Facebook vem utilizando o MongoDB.


\section{NoSQL}

NoSQL é uma classe de sistema de gerenciamento de banco de dados
diferentes dos bancos de dados relacionais tradicionais em que os dados não são
armazenados usando esquemas de tabela fixa. Principalmente o seu propósito é
servir como sistema de banco de dados para aplicações enormes escala web onde
eles superam bancos de dados relacionais tradicionais.

Dentro do aspecto do processamento dos dados, o principal paradigma adotado pelos produtos NoSQL é o MapReduce. Em resumo, tal paradigma divide o processamento em duas etapas: (1) Map, que mapeia e distribui os dados em diversos nós de processamento e armazenamento; e (2) Reduce, que agrega e processa os resultados parciais para gerar um resultado final (ou intermediário para outro processo MapReduce).

Além disso, existem diferentes produtos NoSQL disponíveis no mercado, como por exemplo: Column Store (e.g., Hadoop, Casandra, Amazon SimpleDB, Hypertable), Document Store (e.g., Apache CouchDB, MongoDB), Key-Value/Tuple Store (e.g., Voldemort, Redis, MemcacheDB), Graph Databases (e.g, Neo4J, HyperGraphDB). Assim, serão abordadas implementações existentes para os problemas de distribuição, transações, mineração e a infraestrutura computacional de nuvem.


Em relação a SGBD tradicionais, a distribuição dos dados de forma elástica é inviabilizado pois o modelo de garantia de consistência é fortemente baseado no controle transacional ACID (Atomicity, Consistency, Isolation e Durability). Esse tipo de controle transacional é praticamente inviável quando os dados e o processamento são distribuídos em vários nós. O teorema CAP (Consistency, Availability e Partition tolerance) mostra que somente duas dessas 3 propriedades podem ser garantidas simultaneamente em um ambiente de processamento distribuído de grande porte. A partir desse teorema, os produtos NoSQL utilizam o paradigma BASE (Basically Available, Soft-state Eventually consistency) para o controle de consistência, o que consequentemente traz uma
sensível diminuição no custo computacional para a garantia de consistência dos dados em relação a SGBD tradicionais \cite{mongodb}.

\section{MongoDB}
MongoDB é um sistema de gerenciamento de banco de dados NoSQL
lançado em 2009. Ele armazena dados como documentos JSON-like com
esquemas dinâmicos (o formato é chamado BSON). MongoDB tem seu
foco voltado para quatro coisas: flexibilidade, potência, velocidade e
facilidade de uso. Ele suporta servidores replicados e indexação e oferece
drivers para várias linguagens de programação.


MongoDB aceita dados com tamanho máximo de 16 MB, a distribuição é horizontais escalável, oferece suporte a partilha, foi escrito em C ++. MongoDB é um produto gratuito, nao têm suporte a compressão de dados, usa funções para as operações de adição de novos registros, atualização e exclusão de registros existentes. 

No MongoDB o dancos de dados são armazenados usando coleções. As coleções do MongoDB tem nenhuma restrição em relação aos dados armazenados nele. As coleções não tem uma estrutura fixa. Os campos podem ter diferentes tipos de dados.

A linguagem utilizada pelo MongoDB para consultas avançadas, são funções de
retorno de chamada. Para o mongo shell, a linguagem utilizada para
funções de declaração é JavaScript. Estas funções são usadas para
manipular dados retornados pelas consultas básicas. Pode-se percorrer
os resultados das consultas e aplicar funções diferentes.

A criação da coleção é feita sobre a primeira
inserção. Por causa da flexibilidade do conjunto de MongoDB, a estrutura não tem de ser fixo, sendo assim não há dependências entre as coleções facilitando a exclusão de qualquer tabela e os índices atribuídos a ela \cite{boicea2012mongodb}.
\subsection{Inserir}

A inserção de um novo registro é feito usando
funções e objetos BSON. As funções usadas para inserir novos
dados são 'insert' e 'save'. Ao inserir um novo objeto em uma
coleção MongoDB, um novo campo é adicionado automaticamente
ao seu novo objeto. Esse campo é chamado de '\_id' e é unico, é usado como um índice. Por exemplo, se você quiser adicionar o
mesmo usuário na coleção 'user' do banco de dados MongoDB,
você tem que escrever no shell mongo algo como isto:

\begin{multicols}{2}
	 \begin{verbatim}
	db.users.insert (
	{  
	'id_usuario': 1,
	'primeiro_nome': 'Rafael',
	'ultimo_nome': 'Viana'
	}
	);
	\end{verbatim}

	\begin{verbatim}
	db.users.save (
	{  
	'id_usuario': 1,
	'primeiro_nome': 'Rafael',
	'ultimo_nome': 'Viana'
	}
	);
	\end{verbatim}
	
\end{multicols}
\subsection{Buscar}  
   No MongoDB, você pode recuperar resultados usando a função "find”.
   Por exemplo, supondo que temos uma coleção de usuários
   nomeados “usuarios”. Se você quiser recuperar todos os itens desta
   coleção você pode usar algo como isto:
\begin{verbatim}
	  db.usuarios.find(); 
\end{verbatim}
 Se você quiser apenas para recuperar alguns dos itens na coleção, por
 exemplo, apenas os usuários que tem o primeiro nome “Rafael” você usa
 a função de “find” com um objeto extra: 
\begin{verbatim}
	  db.usuarios.find({primeiro_nome: 'Rafael'});
\end{verbatim}
	 
	 
\subsection{Excluir}  
Se você quiser remover apenas algumas entradas, como os que
têm a 'usuario\_id' maior do que 10, então você deve escrever algo assim: 
\begin{verbatim}
	  db.users.remove ({ 'id_usuario': { '$ GT': 10}}); 
\end{verbatim}
Se você quiser remover todos itens da coleção pode usar a função "Drop", então deve escrever algo assim:
\begin{verbatim}
  db.usuarios.drop();
\end{verbatim}


\subsection{Atualizar}  
Em MongoDB pode facilmente atualizar uma linha ou várias
linhas. O comando se você deseja atualizar toda a coleção é:
\begin{verbatim}
  db.users.update ({},{ '$ Set': { 'primeiro_nome': 'Luiz'}}); 
\end{verbatim}
O primeiro objeto é usado para selecionar as entradas que você deseja atualizar. Neste
caso, a consulta retorna todas as entradas da coleção. Se você quiser atualizar apenas as entradas que o 'user\_id' maior do que 100, a função de atualização parece com isso:
\begin{verbatim}
	  db.users.update ({ 'usuario_id': { '$ gt': 100}},
	    	{ '$ Set': { 'primeiro_nome': 'Luiz'}}); 
\end{verbatim}



\section{Conclusão}
	O MongoDB, tem o foco em escalabilidade e facilidade, e é um excelente complemento para o modelo relacional, no qual estão sendo aplicado em situações para as quais não foram desenvolvidos. A escolha por um modelo não relacional deve estar diretamente relacionada à compatibilidade da sua modelagem com uma ou outra categoria, adiquirindo performace em suas aplicações, que possuem o foco em altos fluxos de dados.
	
	
 
\bibliographystyle{sbc}
\bibliography{sbc-template}

\end{document}
